\let\negmedspace\undefined
\let\negthickspace\undefined
\documentclass[journal,12pt,twocolumn]{IEEEtran}
%\documentclass[conference]{IEEEtran}
%\IEEEoverridecommandlockouts
% The preceding line is only needed to identify funding in the first footnote. If that is unneeded, please comment it out.
\usepackage{cite}
\usepackage{amssymb,amsfonts,amsthm,amsmath}
\usepackage{algorithmic}
\usepackage{graphicx}
\usepackage{textcomp}
\usepackage{xcolor}
\usepackage{txfonts}
\usepackage{listings}
\usepackage{enumitem}
\usepackage{mathtools}
\usepackage{gensymb}

%%
%\usepackage{setspace}
%%\doublespacing
%\singlespacing
%
%%\usepackage{graphicx}
%%\usepackage{amssymb}
%%\usepackage{relsize}
%\usepackage[cmex10]{amsmath}
%%\interdisplaylinepenalty=2500
%%\savesymbol{iint}
%%\usepackage{txfonts}
%%\restoresymbol{TXF}{iint}
%%\usepackage{wasysym}
%\usepackage{amsthm}
%\usepackage{mathrsfs}
%\usepackage{txfonts}
%%\usepackage{stfloats}
%%\usepackage{cite}
%%\usepackage{cases}
%%\usepackage{subfig}
%%\usepackage{xtab}
%%\usepackage{multirow}
%%\usepackage{algorithm}
%%\usepackage{algpseudocode}
%%\usepackage{tikz}
%%\usepackage{circuitikz}
%%\usepackage{verbatim}
\usepackage{hyperref}
%%\usepackage{stmaryrd}
%%\usepackage{tkz-euclide} % loads  TikZ and tkz-base
%%\usetkzobj{all}
    \usepackage{color}                                            %%
    \usepackage{array}                                            %%
    \usepackage{longtable}                                        %%
    \usepackage{calc}                                             %%
    \usepackage{multirow}                                         %%
    \usepackage{hhline}                                           %%
    \usepackage{ifthen}                                           %%
%  %optionally (for landscape tables embedded in another document): %%
%    \usepackage{lscape}     
%%\usepackage{multicol}
%\usepackage{chngcntr}
%\usepackage{enumerate}

%\usepackage{wasysym}
%\newcounter{MYtempeqncnt}
\DeclareMathOperator*{\Res}{Res}
%\renewcommand{\baselinestretch}{2}
\renewcommand\thesection{\arabic{section}}
\renewcommand\thesubsection{\thesection.\arabic{subsection}}
\renewcommand\thesubsubsection{\thesubsection.\arabic{subsubsection}}

\renewcommand\thesectiondis{\arabic{section}}
\renewcommand\thesubsectiondis{\thesectiondis.\arabic{subsection}}
\renewcommand\thesubsubsectiondis{\thesubsectiondis.\arabic{subsubsection}}

% correct bad hyphenation here
\hyphenation{op-tical net-works semi-conduc-tor}
\def\inputGnumericTable{}                                 %%

\lstset{
language=tex,
frame=single, 
breaklines=true
}

\begin{document}
%


\newtheorem{theorem}{Theorem}[section]
\newtheorem{problem}{Problem}
\newtheorem{proposition}{Proposition}[section]
\newtheorem{lemma}{Lemma}[section]
\newtheorem{corollary}[theorem]{Corollary}
\newtheorem{example}{Example}[section]
\newtheorem{definition}[problem]{Definition}
%\newtheorem{thm}{Theorem}[section] 
%\newtheorem{defn}[thm]{Definition}
%\newtheorem{algorithm}{Algorithm}[section]
%\newtheorem{cor}{Corollary}
\newcommand{\BEQA}{\begin{eqnarray}}
\newcommand{\EEQA}{\end{eqnarray}}
\newcommand{\define}{\stackrel{\triangle}{=}}

\bibliographystyle{IEEEtran}
%\bibliographystyle{ieeetr}


\providecommand{\mbf}{\mathbf}
\providecommand{\pr}[1]{\ensuremath{\Pr\left(#1\right)}}
\providecommand{\re}[1]{\ensuremath{\text{Re}\left(#1\right)}}
\providecommand{\im}[1]{\ensuremath{\text{Im}\left(#1\right)}}
\providecommand{\qfunc}[1]{\ensuremath{Q\left(#1\right)}}
\providecommand{\sbrak}[1]{\ensuremath{{}\left[#1\right]}}
\providecommand{\lsbrak}[1]{\ensuremath{{}\left[#1\right.}}
\providecommand{\rsbrak}[1]{\ensuremath{{}\left.#1\right]}}
\providecommand{\brak}[1]{\ensuremath{\left(#1\right)}}
\providecommand{\lbrak}[1]{\ensuremath{\left(#1\right.}}
\providecommand{\rbrak}[1]{\ensuremath{\left.#1\right)}}
\providecommand{\cbrak}[1]{\ensuremath{\left\{#1\right\}}}
\providecommand{\lcbrak}[1]{\ensuremath{\left\{#1\right.}}
\providecommand{\rcbrak}[1]{\ensuremath{\left.#1\right\}}}
\theoremstyle{remark}
\newtheorem{rem}{Remark}
\newcommand{\sgn}{\mathop{\mathrm{sgn}}}
\providecommand{\abs}[1]{\left\vert#1\right\vert}
\providecommand{\res}[1]{\Res\displaylimits_{#1}} 
\providecommand{\norm}[1]{\left\lVert#1\right\rVert}
%\providecommand{\norm}[1]{\lVert#1\rVert}
\providecommand{\mtx}[1]{\mathbf{#1}}
\providecommand{\mean}[1]{E\left[ #1 \right]}
\providecommand{\fourier}{\overset{\mathcal{F}}{ \rightleftharpoons}}
%\providecommand{\hilbert}{\overset{\mathcal{H}}{ \rightleftharpoons}}
\providecommand{\system}{\overset{\mathcal{H}}{ \longleftrightarrow}}
	%\newcommand{\solution}[2]{\textbf{Solution:}{#1}}
\newcommand{\solution}{\noindent \textbf{Solution: }}
\newcommand{\cosec}{\,\text{cosec}\,}
\providecommand{\dec}[2]{\ensuremath{\overset{#1}{\underset{#2}{\gtrless}}}}
\newcommand{\myvec}[1]{\ensuremath{\begin{pmatrix}#1\end{pmatrix}}}
\newcommand{\mydet}[1]{\ensuremath{\begin{vmatrix}#1\end{vmatrix}}}
	\newcommand*{\permcomb}[4][0mu]{{{}^{#3}\mkern#1#2_{#4}}}
\newcommand*{\perm}[1][-3mu]{\permcomb[#1]{P}}
\newcommand*{\comb}[1][-1mu]{\permcomb[#1]{C}}

%\numberwithin{equation}{section}
\numberwithin{equation}{subsection}
%\numberwithin{problem}{section}
%\numberwithin{definition}{section}
\makeatletter
\@addtoreset{figure}{problem}
\makeatother

\let\StandardTheFigure\thefigure
\let\vec\mathbf
\let\j\jmath
%\renewcommand{\thefigure}{\theproblem.\arabic{figure}}
\renewcommand{\thefigure}{\theproblem}
%\setlist[enumerate,1]{before=\renewcommand\theequation{\theenumi.\arabic{equation}}
%\counterwithin{equation}{enumi}


%\renewcommand{\theequation}{\arabic{subsection}.\arabic{equation}}

\def\putbox#1#2#3{\makebox[0in][l]{\makebox[#1][l]{}\raisebox{\baselineskip}[0in][0in]{\raisebox{#2}[0in][0in]{#3}}}}
     \def\rightbox#1{\makebox[0in][r]{#1}}
     \def\centbox#1{\makebox[0in]{#1}}
     \def\topbox#1{\raisebox{-\baselineskip}[0in][0in]{#1}}
     \def\midbox#1{\raisebox{-0.5\baselineskip}[0in][0in]{#1}}

\vspace{3cm}

\title{
%	\logo{
	MATRICES
%	}
}
\author{
	Lanka NIkhil Manikanta\\
	EE22MTECH11016
	%<-this % stops a space
%\thanks{}}
}

\maketitle

\newpage

\tableofcontents

\bigskip

\renewcommand{\thefigure}{\theenumi}
\renewcommand{\thetable}{\theenumi}

\section{Least Squares}
Consider the lines 
\begin{align}
    L_1\colon \vec{x}=\myvec{1\\2\\1}+\lambda_1\myvec{1\\-1\\1}\label{eq:pseudo:1}\\
    L_2\colon \vec{x}=\myvec{2\\-1\\-1}+\lambda_2\myvec{2\\1\\2}\label{eq:pseudo:2}
\end{align}

%\renewcommand{\theequation}{\theenumi}
\begin{enumerate}[label=\thesection.\arabic*.,ref=\thesection.\theenumi]
\numberwithin{equation}{enumi}

\item If the two lines intersect, show that 
\begin{align}
\vec{M}\myvec{\lambda_1 \\ -\lambda_2} = \vec{x}_2 - \vec{x}_1
\label{eq:pseudo_intersect}
\end{align}
%
where 
\begin{align}
\label{eq:pseudo_xm}
\vec{x}_1 = \myvec{1\\2\\1},
\vec{x}_2 = \myvec{2\\-1\\-1},
\vec{m}_1 = \myvec{1\\-1\\1},
\vec{m}_2 = \myvec{2\\1\\2}.
\\
\vec{M} =\myvec{\vec{m}_1 & \vec{m}_2}
\label{eq:pseudo_rect}
\end{align}
\solution\\ From \eqref{eq:pseudo:1} and \eqref{eq:pseudo:2}, we have,
\begin{align*}
    L_1\colon& \vec{x}=\myvec{1+\lambda_1\\2-\lambda_1\\1+\lambda_1}\ \\
    L_2\colon& \vec{x}=\myvec{2+2\lambda_2\\-1+\lambda_2\\-1+2\lambda_2}\ \\
\end{align*}
Given that the lines $L_1$ and $L_2$ intersect. So, they will have a common point on them for some values $\lambda_1~\text{and}~\lambda_2$.
\begin{align*}
& \myvec{1+\lambda_1\\2-\lambda_1\\1+\lambda_1}\ =\myvec{2+2\lambda_2\\-1+\lambda_2\\-1+2\lambda_2}\ \\
\implies & \myvec{1+\lambda_1-(2+2\lambda_2)\\2-\lambda_1-(-1+\lambda_2)\\1+\lambda_1-(-1+2\lambda_2)}\ = \myvec{0\\0\\0}\ \\
\implies & \myvec{\lambda_1-2\lambda_2-1\\-\lambda_1-\lambda_2+3\\\lambda_1-2\lambda_2+2}\ = \myvec{0\\0\\0}\ \\
\implies & \myvec{\lambda_1-2\lambda_2\\ -\lambda_1-\lambda_2\\\lambda_1-2\lambda_2}\ = \myvec{1\\-3\\-2}\ \\
\implies & \myvec{1 & -2\\ -1 & -1\\ 1 & -2}\ \myvec{\lambda_1\\ \lambda_2}\ = \myvec{1\\-3\\-2}\ \\
\end{align*}
\begin{align}
\implies & \myvec{1 & 2\\ -1 & 1\\ 1 & 2}\ \myvec{\lambda_1\\ -\lambda_2}\ = \myvec{1\\-3\\-2}\ \label{eq:pseudo_sol1_1} 
\end{align}
From \eqref{eq:pseudo_xm}, we have 
\begin{align}
& \vec{x}_2-\vec{x}_1= \myvec{1\\-3\\-2}\ \label{eq:pseudo_sol1_2} \\
& \vec{M}=\myvec{1 & 2\\ -1 & 1\\ 1 & 2}\ \label{eq:pseudo_sol1_3}
\end{align}
Substituting, \eqref{eq:pseudo_sol1_2} and \eqref{eq:pseudo_sol1_3} in \eqref{eq:pseudo_sol1_1} , we have 
\begin{align}
\vec{M}\myvec{\lambda_1 \\ -\lambda_2} = \vec{x}_2 - \vec{x}_1 \label{eq:pseudo_sol1_4}
\end{align}
Hence Proved\qed
\item Find the rank of the augmented matrix in 
\eqref{eq:pseudo_intersect}.
	Show that  the lines in \eqref{eq:pseudo:1},\eqref{eq:pseudo:2}
 do not intersect.\\
\solution\\
From \eqref{eq:pseudo_sol1_3}, we know that the augmented matrix will be
\begin{align*}
\vec{(M|x_2-x_1)}&=\myvec{1 & 2 & 1\\ -1 & 1 & -3\\ 1 & 2 &-2} \\
&=\myvec{1 & 2 & 1\\ 0 & 3 & -2\\ 1& 2 &-2}~~~\text{R2= R2+R1}\\
&=\myvec{1 & 2 & 1\\ 0 & 3 & -2\\ 0& 0 &-3}~~~\text{R3= R3-R1}\\
\end{align*}
Since the number of independent rows are 3. the rank of the augmented matrix is \textbf{3}.\\
Let's assume that the lines intersect, then, from 1.1 we have,
\begin{align*}
    L_1\colon& \vec{x}=\myvec{1+\lambda_1\\2-\lambda_1\\1+\lambda_1}\ \\
    L_2\colon& \vec{x}=\myvec{2+2\lambda_2\\-1+\lambda_2\\-1+2\lambda_2}\ \\
\end{align*}
Accordingly,
\begin{align*}
& \myvec{1+\lambda_1\\2-\lambda_1\\1+\lambda_1}\ =\myvec{2+2\lambda_2\\-1+\lambda_2\\-1+2\lambda_2}\ \\
\implies & \myvec{1+\lambda_1-(2+2\lambda_2)\\2-\lambda_1-(-1+\lambda_2)\\1+\lambda_1-(-1+2\lambda_2)}\ = \myvec{0\\0\\0}\ \\
\implies & \myvec{\lambda_1-2\lambda_2-1\\-\lambda_1-\lambda_2+3\\\lambda_1-2\lambda_2+2}\ = \myvec{0\\0\\0}\ \\
\implies & \myvec{1 & -2\\ -1 & -1\\ 1 & -2}\ \myvec{\lambda_1\\ \lambda_2}\ = \myvec{1\\-3\\-2}\ \\
\end{align*}
Let's find the values of $\lambda_1$ and $\lambda_2$ for the above three equations:
\begin{align}
\lambda_1-2\lambda_2=1\label{eq:pseudo_sol_1_2_1}\\
\lambda_1+\lambda_2=3 \label{eq:pseudo_sol_1_2_2}\\
\lambda_1-2\lambda_2=-2\label{eq:pseudo_sol_1_2_3}
\end{align}
On solving the \eqref{eq:pseudo_sol_1_2_1} and \eqref{eq:pseudo_sol_1_2_2}, we get $\lambda_1=\frac{7}{3}$ and $\lambda_2=\frac{2}{3}$. So, we substitute these in \eqref{eq:pseudo_sol_1_2_3} and get,
\begin{align*}
& \lambda_1-2\lambda_2=-2\\
& \frac{7}{3}-2\times\frac{2}{3}=-2\\
& 1\neq-2
\end{align*}
So, the lines do  not intersect with each other.\\
Hence Proved \qed

\item Let 
\begin{align}
\vec{A}=\vec{x}_1 + \lambda_1 \vec{m}_1
\\
\vec{B}=\vec{x}_2 + \lambda_2 \vec{m}_2
\label{eq:pseudo_AB}
\end{align}
be the closest points on $L_1$ and $L_2$ respectively.  Then the shortest distance between two skew lines 
will be the length of line perpendicular to both the lines $L_1$,$L_2$ and passing through $A$ and $B$.
Show that 
\begin{align}
\vec{M}^T\brak{\vec{A}-\vec{B}} = 0
\label{eq:pseudo:MAB}
\end{align}
From \eqref{eq:pseudo_AB} and \eqref{eq:pseudo_rect}
\begin{align*}
    \vec{A-B}&=\vec{x_1-x_2}+\vec{M}\myvec{\lambda_1\\ -\lambda_2}\\
    &=\vec{x_1-x_2}+\vec{x_2-x_1}~~~~~\text{from \eqref{eq:pseudo_sol1_4}}\\
    &=\vec{0}
\end{align*}
Accordingly, we have from \eqref{eq:pseudo_sol1_3}
\begin{align*}
& \vec{M^T}=\myvec{1 & -1 & 1\\2& 1 & 2}\ \\
\implies &\vec{M^T}(A-B)=\myvec{1 & -1 & 1\\2& 1 & 2}\myvec{0 \\ 0 \\ 0}=0
\end{align*}
Hence Proved \qed
\item Show that
\begin{align}
    \vec{M}^T\vec{M}\myvec{\lambda_1\\-\lambda_2} = \vec{M}^T\brak{\vec{x}_2-\vec{x}_1}
\label{eq:pseudo_ls}
\end{align}\\
\solution\\
We know that 
\begin{align*}
& \vec{M}\myvec{\lambda_1\\ -\lambda_2}=\vec{x_2-x_1}\\
& \text{Multiplying both sides of above equation with $\vec{M^T}$}\\
&  \vec{M}^T\vec{M}\myvec{\lambda_1\\-\lambda_2} = \vec{M}^T\brak{\vec{x}_2-\vec{x}_1}
\end{align*}
Hence Proved\qed
\item Obtain $\lambda_1$ and $\lambda_2$.\\
\solution\\ Using the above proved equation,we have,
\begin{equation*}
\vec{M}^T\vec{M}\myvec{\lambda_1\\-\lambda_2} = \vec{M}^T\brak{\vec{x}_2-\vec{x}_1}
\end{equation*}
So, we have
\begin{align*}
& \vec{M^T}=\myvec{1 & -1 & 1\\2& 1 & 2}\\
& \vec{M}=\myvec{1 & 2\\ -1 & 1\\ 1 & 2}\ 
\end{align*}
So, we have,
\begin{align*}
\vec{M}\myvec{\lambda_1\\ -\lambda_2}&=\myvec{1 & 2\\ -1 & 1\\ 1 & 2}\myvec{\lambda_1\\ -\lambda_2}\\
\implies \vec{M}\myvec{\lambda_1\\ -\lambda_2}&=\myvec{\lambda_1-2\lambda_2\\ -\lambda_1-\lambda_2\\\lambda_1-2\lambda_2}
\end{align*}
\begin{align*}
&\vec{M^T}\vec{M}\myvec{\lambda_1\\ -\lambda_2}=\myvec{1 & -1 & 1\\2& 1 & 2}\myvec{\lambda_1-2\lambda_2\\ -\lambda_1-\lambda_2\\\lambda_1-2\lambda_2}\\
& \vec{M^T}\vec{M}\myvec{\lambda_1\\ -\lambda_2}=\myvec{1 & -1 & 1\\2& 1 & 2}\myvec{\lambda_1-2\lambda_2\\ -\lambda_1-\lambda_2\\\lambda_1-2\lambda_2}\\
& \vec{M^T}\vec{M}\myvec{\lambda_1\\ -\lambda_2}=\myvec{3\lambda_1-3\lambda_2\\3\lambda_1-9\lambda_2}
\end{align*}
Now, for $\vec{M}^T\brak{\vec{x}_2-\vec{x}_1}$, we have,
\begin{align*}
\vec{M}^T\brak{\vec{x}_2-\vec{x}_1}&=\myvec{1 & -1 & 1\\2& 1 & 2}\myvec{1\\-3\\-2}=\myvec{2\\-5}
\end{align*}
So, 
\begin{align*}
& \myvec{3\lambda_1-3\lambda_2\\3\lambda_1-9\lambda_2}=\myvec{2\\-5}\\
& \lambda_1=\frac{7}{6}; \lambda_2=\frac{11}{6}
\end{align*}
\item Show that the distance between the lines is 
\begin{align}
    \frac{3}{\sqrt{2}}
\end{align}
\solution\\
Given that,
\begin{align*}
A=x_1+ \lambda_1 m_1\\
B=x_2+ \lambda_2 m_2\\
\end{align*}
Therefore,
\begin{align*}
A=\myvec{1\\2\\1} + \lambda_1 \myvec{1\\-1\\1} \\
B=\myvec{2\\-1\\-1} + \lambda_2\myvec{2\\1\\2}\\
\end{align*}
Now, the shortest distance is given by:
\begin{align*}
 \lvert   \frac{(\overrightarrow{m_1} \times \overrightarrow{m_2})(\overrightarrow{x_2}-\overrightarrow{x_1})}{\lvert\lvert\overrightarrow{m_1}\times \overrightarrow{m_2}\rvert\rvert}\rvert
\end{align*}\label{eq:sht_distance}
\begin{align*}
&  (\overrightarrow{x_2}-\overrightarrow{x_1})\\
&  =\myvec{2\\-1\\-1}-\myvec{2\\1\\2}\\
&  = \myvec{1\\-3\\-2}\\
&  =\hat{i}-3\hat{j}-2\hat{k}\\
\end{align*}
Now,
\begin{align*}
& (\overrightarrow{m_1}\times \overrightarrow{m_2})
\end{align*}
$$
\begin{vmatrix}

\hat{i} & \hat{j} & \hat{k} \\

 1 & -1 & 1 \\

 2 & 1 & 2

\end{vmatrix}
$$

\begin{align*}
& =(-3\hat{i} + 0\hat{j} + 3\hat{k})\\
& =\sqrt[]{9+0+9}\\
& =\sqrt[]{18}\\
& =(\overrightarrow{m_1}\times \overrightarrow{m_2})=3\sqrt[]{2}
\end{align*}

Now, putting these values in \eqref{eq:sht_distance} we get,

\begin{align*}
& \frac{(-3\hat{i} +0\hat{j} +3\hat{k}).(\hat{i}-3\hat{j}-2\hat{k})}{3\sqrt[]{2}}\\
& =|\frac{-9}{3\sqrt[•]{2}}|\\
& =\frac{3}{\sqrt{2}}
\end{align*}
\end{enumerate}

\section{Singular Value Decomposition}
\begin{enumerate}[label=\thesection.\arabic*.,ref=\thesection.\theenumi]
\numberwithin{equation}{enumi}

\item Find $\vec{M}^T\vec{M}$ and $\vec{M}\vec{M}^T$.\\
\solution\\
We know that
$$M=
\begin{pmatrix}
1 & 2 \\
-1 & 1 \\
1 & 2
\end{pmatrix}
$$
$$M^T=
\begin{pmatrix}
1 & -1 & 1\\
2 & 1 & 2 
\end{pmatrix}
$$
Now $M^T M$ is
$$
\begin{pmatrix}
1 & -1 & 1\\
2 & 1 & 2
\end{pmatrix} \times
\begin{pmatrix}
1 & 2\\
-1 & 1\\
1 & 2
\end{pmatrix}
$$
$$
=\begin{pmatrix}
1+1+1 & 2-1+2\\
2-1+2 & 4+1+4
\end{pmatrix}
$$
$$
=\begin{pmatrix}
3 & 3\\
3 & 9
\end{pmatrix}
$$
Now, $MM^T$ is\\
$$
\begin{pmatrix}
1 & 2\\
-1 & 1\\
1 & 2
\end{pmatrix} \times
\begin{pmatrix}
1 & -1 & 1\\
2 & 1 & 2
\end{pmatrix} 
$$
$$
=\begin{pmatrix}
1+4 & -1+2 & 1+4\\
-1+2 & 1+1 &-1+2\\
1+4 & -1+2 & 1+4
\end{pmatrix}
$$
$$
=\begin{pmatrix}
5 & 1 & 5\\
1 & 2 & 1\\
5 & 1 & 5
\end{pmatrix}
$$
\item Obtain the eigen decomposition 
\begin{align}
\vec{M}^T\vec{M} = \vec{P}_1\vec{D}_1\vec{P}_1^T
\end{align}
and 
\begin{align}
\vec{M}\vec{M}^T = \vec{P}_2\vec{D}_2\vec{P}_2^T
\end{align}
\solution
From above we know that,
$$
M^TM=\begin{pmatrix}
3 & 3\\
3 & 9
\end{pmatrix}
$$
For finding Eigen values we have to use:\\
$A-\lambda I=0$\\
where $A=M^TM$ and $I$ is $2\times 2$ identity matrix.
Hence by putting their values we get,
$$
\begin{pmatrix}
3 & 3\\
3 & 9
\end{pmatrix}
-\lambda
\begin{pmatrix}
1 & 0\\
0 & 1
\end{pmatrix}
=0
$$
$$
\begin{pmatrix}
3-\lambda & 3\\
3 & 9-\lambda
\end{pmatrix}
=0
$$
By further solving we get,
\begin{align*}
& (3-\lambda)(9-\lambda)-9=0\\
& =27-3\lambda-9\lambda+\lambda^2=0\\
& =\lambda^2-12\lambda+18=0\\
& =\lambda_1=6+3\sqrt{2} & \lambda_2=6-3\sqrt{2}
\end{align*}
Now, we need to find Eigen Vectors\\
Eigen vector $v_1$ for $\lambda_1$ ,\\
where $\lambda_1 =6+3\sqrt{2}$\\
And we know that,
$A.v_1 = \lambda_1.v_1$\\
Hence, $(A-\lambda_1).v_1=0$\\
where $A=M_TM $\\
Therefore by putting values we get,\\
$$
\begin{pmatrix}
3-\lambda_1 & 3\\
3 & 9-\lambda_1
\end{pmatrix}
.v_1=0
$$
and by further calculating we get $v_1$ as\\
\begin{align*}
=\myvec{\sqrt{2}-1\\1}
\end{align*}
Now for $v_2$ we use value of $\lambda_2 (6-3\sqrt{2})$ and by following the same procedure we get $v_2$ as,
\begin{align*}
=\myvec{-1-\sqrt{2}\\1}
\end{align*}
Now, $P_1$ will be \\
$$
P_1 =\begin{pmatrix}
-1-\sqrt{2} & \sqrt{2}-1\\
1 & 1
\end{pmatrix}
$$
The diagonal matrix of eigen values $D_1$ will be:
$$
D_1=\begin{pmatrix}
6+3\sqrt{2} & 0\\
0 & 6-3\sqrt{2}
\end{pmatrix}
$$  
Finally we have to calculate inverse of $P_1$.\\
Hence, $P_1^{-1}$ is:
$$
\begin{pmatrix}
\frac{-1}{2\sqrt{2}}  && \frac{(2-\sqrt{2})}{4}\\
\frac{1}{2\sqrt{2}}  && \frac{(2+\sqrt{2})}{4}\\
\end{pmatrix}
 $$
  Now we need to compute $P_1D_1P_1^{-1}$
  Therefore by putting all these values we get,
  $$
\begin{pmatrix}
-1-\sqrt{2} & \sqrt{2}-1\\
1 & 1
\end{pmatrix}
  \times
\begin{pmatrix}
6+3\sqrt{2} & 0\\
0 & 6-3\sqrt{2}
\end{pmatrix}
\times
$$
$$
\begin{pmatrix}
\frac{-1}{2\sqrt{2}}  && \frac{(2-\sqrt{2})}{4}\\
\frac{1}{2\sqrt{2}}  && \frac{(2+\sqrt{2})}{4}\\
\end{pmatrix}
$$
By multiplying all the three matrices we get,
$$
\begin{pmatrix}
3 & 3\\
3 & 9
\end{pmatrix} 
= M_TM
$$
\\
From above we know that,
$$
MM_T=\begin{pmatrix}
5 & 1& 5\\
1 & 2& 1\\
5 &1 & 5
\end{pmatrix}
$$
For finding Eigen values we have to use:\\
$A-\lambda I=0$\\
where $A=M^TM$ and $I$ is $3\times 3$ identity matrix.
Hence by putting their values we get,
$$
\begin{pmatrix}
5 & 1 & 5\\
1 & 2 & 5\\
5 & 1 & 5
\end{pmatrix}
-\lambda
\begin{pmatrix}
1 & 0 & 0\\
0 & 1 & 0\\
0 & 0 & 1
\end{pmatrix}
=0
$$
$$
\begin{pmatrix}
5-\lambda & 1 & 5\\
1 & 2-\lambda & 1\\
5&1&5-\lambda
\end{pmatrix}
=0
$$
By further solving we get,
\begin{align*}
& (5-\lambda)[(2-\lambda)(5-\lambda)]-1[1(5-\lambda)-5]+\\
& 5[1-(10-5\lambda)]=0\\
& =-\lambda^{3}+12\lambda^{2}-18\lambda=0\\
& \lambda_1=3(2+2\sqrt{2}), \\          
& \lambda_2=-3(\sqrt{2}-2),\\
& \lambda_3=0
\end{align*}

Now, we need to find Eigen Vectors\\
Eigen vector $v_1$ for $\lambda_1$ ,\\
where $\lambda_1 =3(2+2\sqrt{2})$\\
And we know that,
$A.v_1 = \lambda_1.v_1$\\
Hence, $(A-\lambda_1).v_1=0$\\
where $A=MM_T $\\
Therefore by putting values we get,\\
$$
\begin{pmatrix}
5-\lambda_1 & 1 & 5\\
1 & 2-\lambda_1 & 1\\
5 & 1 & 5-\lambda_1
\end{pmatrix}
.v_1=0
$$
and by further calculating we get $v_1$ as\\
\begin{align*}
=\myvec{1\\-4+3\sqrt{2}\\1}
\end{align*}
By putting the value of $\lambda_2(-3(\sqrt{2}-2)) $ we get $v_2$ as:
$$
\begin{pmatrix}
5-\lambda_2 & 1 & 5\\
1 & 2-\lambda_2 & 1\\
5 & 1 & 5-\lambda_2
\end{pmatrix}
.v_1=0
$$
and by further calculating we get $v_2$ as\\
\begin{align*}
=\myvec{1\\-4-3\sqrt{2}\\1}
\end{align*}
By putting the value of $\lambda_2(0) $ we get $v_3$ as:
$$
\begin{pmatrix}
5-\lambda_2 & 1 & 5\\
1 & 2-\lambda_2 & 1\\
5 & 1 & 5-\lambda_2
\end{pmatrix}
.v_1=0
$$
and by further calculating we get $v_3$ as\\
\begin{align*}
=\myvec{-1\\0\\1}
\end{align*}
Now $P_2$ will be
$$
P_2=\begin{pmatrix}
-1 &1& 1\\
0 & -4-3\sqrt{2} & 3\sqrt{2}-4\\
1 & 1 & 1
\end{pmatrix} 
$$
The diagonal matrix of eigen values $D_2$ will be:
$$
D_2=\begin{pmatrix}
0 & 0 & 0\\
3(2+2\sqrt{2}) & 0 & 0\\
0 & -3(\sqrt{2}-2) & 0
\end{pmatrix}
$$
And now $P_2^{-1}$ which is inverse of $P_2$ will be:
$$
\begin{pmatrix}
\frac{1}{2} & 0 & \frac{-1}{2}\\
\frac{1}{12}(3-2\sqrt{2} & \frac{-1}{6\sqrt{2}} & \frac{1}{12}(3-2\sqrt{2}))\\
\frac{1}{12}(3+2\sqrt{2} & \frac{1}{6\sqrt{2}} & \frac{1}{12}(3+2\sqrt{2}))\\
\end{pmatrix}
$$
Now we need to compute $P_2D_2P_2^{-1}$.\\
Hence, By putting all the values we get, 
$$
\begin{pmatrix}
5&1&5\\
1&2&1\\
5&1&5 
\end{pmatrix} =MM_T
$$
Hence Proved
\item Perform the $QR$ decompositions
\begin{align}
\vec{P}_1 = \vec{U}\vec{R}_1\\
\vec{P}_2 = \vec{V}\vec{R}_2
\end{align}
\solution\\
From 2.2 , we know that 
$$
P_1 =\begin{pmatrix}
-1-\sqrt{2} & \sqrt{2}-1\\
1 & 1
\end{pmatrix}
$$
$$
P_2=\begin{pmatrix}
-1 &1& 1\\
0 & -4-3\sqrt{2} & 3\sqrt{2}-4\\
1 & 1 & 1
\end{pmatrix} 
$$
Let us first solve for $\vec{P_1}$\\
\begin{align*}
& \vec{P_1}=\myvec{-2.4142 & 0.4142\\1 & 1}\\
& q_1{'}=a_1=\myvec{-2.4142 \\1 }\\
& r_{11}=||q_1{'}||=\sqrt{(-2.4142)^2+1^2}=\sqrt{6.8284}\\\
& r_{11}=2.6131\\
& q_1=\frac{1}{||q_1{'}|}\times q_1{'}=\frac{1}{2.6131}\myvec{-2.4142 \\1}\\
& q_1=\myvec{-0.9239 \\0.3827}\\
& r_{12}=q_1^T.a_2=\myvec{-0.9239 & 0.3827} \myvec{0.4142\\ 1}=0\\
&q_2'=a_2-r_{12}\times q_1=\myvec{0.4142\\ 1}\\
& r_{22}=||q_2'||=\sqrt{0.4142^2+1^2}=1.0824\\
& q_2=\frac{1}{||q_2'||}\times q_2'=\frac{1}{1.0824}\myvec{0.4142\\ 1}=\myvec{0.3827\\ 0.9239}
\end{align*}
\begin{align*}
&\vec{Q}=\myvec{q_1 & q_2}=\myvec{-0.9239 & 0.3827\\0.3827 & 0.9239}\\
& \vec{R}=\myvec{r_{11} & r_{12}\\ 0 & r_{22}}=\myvec{2.6131 & 0\\ 0 & 1.0824}
\end{align*}
Again, upon mulitplying $\vec{Q}\times\vec{R}=\vec{P_1}$
Now, we solve for $\vec{P_2}$
\begin{align*}
& \vec{P_2}=\myvec{-1 & 1 & 1\\0 & -8.42436 & 0.2426\\1 & 1 & 1}\\
& q_1{'}=a_1=\myvec{-1\\0 \\1 }\\
& r_{11}=||q_1{'}||=\sqrt{(1)^2+0^2+1^2}=\sqrt{2}\\\
& r_{11}=1.4142\\
& q_1=\frac{1}{||q_1{'}|}\times q_1{'}=\frac{1}{1.4142}\myvec{-1\\0 \\1 }\\
& q_1=\myvec{-0.7071\\0 \\0.7071 }\\
& r_{12}=q_1^T.a_2=\myvec{-0.7071\\0 \\0.7071 } \myvec{1\\-8.2426\\ 1}=0\\
&q_2'=a_2-r_{12}\times q_1=\myvec{1\\-8.2426\\ 1}\\
& r_{22}=||q_2'||=\sqrt{1^2+(-8.2426)^2+1^2}=8.3631\\
& q_2=\frac{1}{||q_2'||}\times q_2'=\frac{1}{8.3631}\myvec{1 \\-8.2426\\ 1}=\myvec{0.1196\\ -0.9856\\0.1196}\\
& r_{13}=q_1^T.a_3=\myvec{-0.7071 & 0 & 0.7071 } \myvec{1\\0.2426\\ 1}=0\\
& r_{23}=q_2^T.a_3=\myvec{0.1196& -0.9856& 0.1196} \myvec{1\\0.2426\\ 1}=0\\
&q_3'=a_3-r_{13}.q_1-r_{23}.q_2=\myvec{1\\0.2426\\ 1}-0-0\\
&q_3'=\myvec{1\\0.2426\\ 1}\\
& r_{33}=||q_3'||=\sqrt{1^2+(0.2426)^2+1^2}=1.4349\\
& q_3=\frac{1}{||q_3'||}\times q_3'=\frac{1}{1.4349}\myvec{1 \\0.2426\\ 1}=\myvec{0.6969\\ 0.1691\\0.6969}
\end{align*}
\begin{align*}
&\vec{Q}=\myvec{q_1 & q_2 &q_3}=\myvec{-0.7071 & 0.1196 & 0.6969 \\ 0 & -0.9856 & 0.1691\\ 0.7071 & 0.1196 & 0.6969}\\
& \vec{R}=\myvec{r_{11} & r_{12} & r_{13}\\ 0 & r_{22} & r_{23}\\0 & 0& r_{33}}=\myvec{1.4142 & 0 & 0\\ 0 & 8.3631 & 0\\0 & 0 & 1.4349}
\end{align*}
Again, upon mulitplying $\vec{Q}\times\vec{R}=\vec{P_2}$.\qed
\item The singular value decomposition is the given by
\begin{align}
\vec{M} = \vec{U} \Sigma \vec{V}^T,
\end{align}
where $\Sigma$ has the same shape as $\vec{M}$ and
\begin{align}
\Sigma = \myvec{\vec{D}_1 & \vec{0} \\ \vec{0} & \vec{0}}
\end{align}
\solution\\
From the solution of 2.2, we already know that 
$$
M^TM=\begin{pmatrix}
3 & 3\\
3 & 9
\end{pmatrix}
$$
For finding Eigen values we use $A-\lambda I=0$. So, we have the eigen values as\\
$\lambda_1=6+3\sqrt{2}$, $\lambda_2=6-3\sqrt{2}$.
Accordingly, we have eigen vectors as $e_1=(\sqrt{2})-1,1)$ and $e_2=(-1-\sqrt{2},1)$ respectively. Or we use $e_1=(0.41421356,1)$ and $e_2=(-2.41421356,1)$ respectively. The $||e_1||=1.0823922$ and $||e_2||=2.61312593$\\
So, the normalizing these gives 
\begin{align*}
&v_1=\myvec{\frac{0.41421356}{1.0823922},\frac{1}{1.0823922}}\\
&v_2=\myvec{\frac{-2.41421356}{2.61312593},\frac{1}{2.61312593}}
\end{align*}
So,
\begin{align*}
\sum&=\myvec{\sqrt{10.24264069}	& 0\\0 & \sqrt{1.75735931}\\ 0 & 0} \\
&=\myvec{3.2004158& 0\\0 & 1.3256543\\ 0 & 0}\\
\vec{V}&=\myvec{v_1,v_2}=\myvec{0.38268343 & -0.92387953\\ 0.92387953	& 0.38268343}\\
\end{align*} 
Accordingly,$U$, will be $u_i=\frac{1}{\sigma_i}M.v_i$
\begin{align*}
\vec{U}=\myvec{	0.69692342	& -0.11957316 & -0.70710678\\	
	0.16910198	& 0.98559856 & 0	\\
	0.69692342	&-0.11957316	& 0.70710678}
\end{align*}
Thus, $\vec{M}=\vec{U}\sum \vec{V^T}$
Now, for $\vec{MM^T}$, we have
 $$
MM_T=\begin{pmatrix}
5 & 1& 5\\
1 & 2& 1\\
5 &1 & 5
\end{pmatrix}
$$
For finding Eigen values we use $A-\lambda I=0$. So, we have the eigen values as\\
$\lambda_1=3(2+2\sqrt{2})$,\\ $\lambda_2=-3(\sqrt{2}-2)$,\\$\lambda_3=0$.\\ Accordingly, we have\\ $e_1=\myvec{1 & 0.24264069 & 1}$,\\$ e_2=\myvec{1	&-8.24264069 & 1}$,\\ $e_3=\myvec{-1 & 0 & 1}$\\ respectively as already calculated in 2.2. So, \\
$||e_1||=1.43487787$,\\
$||e_2||=8.3630811$,\\
$||e_3||=1.41421356$\\
Thus, the normalizing gives 
\begin{align*}
&u_1=\myvec{0.69692343 & 0.16910198 & 0.69692343}\\
&u_2=\myvec{0.11957316,-0.98559856,0.11957316}\\
&u_3=\myvec{-0.70710678 &0 & 0.70710678}
\end{align*}
So,
\begin{align*}
\sum&=\myvec{\sqrt{10.24264069}	& 0\\0 & \sqrt{1.75735931}\\ 0 & 0} \\
&=\myvec{3.2004158& 0\\0 & 1.3256543\\ 0 & 0}\\
\end{align*}
\begin{align*}
\sum&=\myvec{\sqrt{10.24264069}	& 0\\0 & \sqrt{1.75735931}\\ 0 & 0} \\
&=\myvec{3.2004158& 0\\0 & 1.3256543\\ 0 & 0}\\
\vec{V}&=\myvec{v_1,v_2}=\myvec{0.38268343 & 0.92387953\\ 0.92387953	& -0.38268343}\\
\end{align*}
Accordingly,$U$, will be $u_i=\frac{1}{\sigma_i}M.v_i$
\begin{align*}
\vec{U}=\myvec{	0.69692342	& 0.11957316 & -0.70710678\\	
	0.16910198	& -0.98559856 & 0	\\
	0.69692342	&-0.11957316	& 0.70710678}
\end{align*}

\item Let
\begin{align*}
	\vec{b} &= 
	\vec{x}_2-\vec{x}_1
	\\
	\vec{y} &= \myvec{\lambda_1 \\ - \lambda_2}
\end{align*}
	%\eqref{eq:pseudo_mat_eq} 
\eqref{eq:pseudo_intersect}
can then be expressed as
\begin{align*}
\vec{U} \Sigma \vec{V}^T \vec{y} &= \vec{b}\\
\implies \vec{y} & = \vec{V}\Sigma^{-1} \vec{U}^T \vec{b}
\end{align*}

%
where $\vec{\Sigma}^{-1}$ is obtained by inverting  only the non-zero elements of $\vec{\Sigma}$.
\end{enumerate}
\solution
From the SVD-1 take U and V
\\
\begin{align*}
{\Sigma}^{-1}=\myvec{0.31245939& 0\\0 & 0.754344477\\ 0 & 0}
\end{align*}
\\
\begin{align*}
y= \vec{V}\Sigma^{-1} \vec{U}^T \vec{b}
\end{align*}
\begin{align*}
\vec{y}=\myvec{	0.16666667	& -0.666666667 & 0.16666667\\	
	0.16666667	& 0.33333332 & 0.16666667  }*b
\end{align*}
\begin{align*}
y=\myvec{	0.16666667	\\ -0.66666667}
\end{align*}

\\
From the SVD -2 take U and V
\\
\begin{align}
\\{\Sigma}^{-1}=\myvec{0.31245939& 0\\0 & 0.754344477\\ 0 & 0}
\end{align}
\begin{align*}
y= \vec{V}\Sigma^{-1} \vec{U}^T \vec{b}
\end{align*}
\begin{align*}
\vec{y}=\myvec{	0.11957316	& 0.69692343 & 0\\	
	0.28867513	& -0.28867513 & 0  }*b
\end{align*}
\begin{align*}
y=\myvec{	-1.97119713	\\ 1.15470052}
\end{align*}

\newpage

\end{document}
%
%
